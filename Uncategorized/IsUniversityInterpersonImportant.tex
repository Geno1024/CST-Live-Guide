\section{大学里的人际关系重要吗?}\label{sec:Uncategorized/IsUniversityInterpersonImportant}
%Gloomy:
%大学里的人际关系重要吗?(是否会影响到未来的一些东西呢)
%
%Gloomy:
%因为听一些人说(其实毕业了屁影响没有)
%
%Gloomy:
%但又有一些说(还挺重要的)
    有人说,大学里的人际关系在毕业了之后什么影响都没有,因此大学里的人际关系并不重要;但是也有一些人说挺重要的。那,有人就在问,大学里的人际关系重要吗?

    很明显,这是一个人文社会类的问题,而不是一个理工科学类的问题。人文社会类问题相对于理工科学类问题有一个显著特点就是,它通常情况下并不存在一个(在一定范围与环境内)正确的答案,各人都可以根据自己的实际经历进行回答,只要言之有“理”就是一个答案,而且这个“理”会与社会变迁、区域地理、时代改革甚至地位身份、自身利益等而变化。因此,虽然这种“是”或“否”式的判断题的回答要么是“是的,大学里的人际关系重要”要么是“并不,大学里的人际关系并不重要”,或者模糊的“大学里的人际关系可以很重要也可以很不重要”,但是说出其背后的理由才是大家想要的。

    也因此,在这里 Geno 只能根据自身的经历进行主观的回答,并以个人努力试图贴近实际。但是需要注意的是,个人观点并不一定可以作为行事指导,只是自身经历的表述,这些观点不一定正确,也不一定错误,只是作为一个“存在这样一种观点”的佐证参考。

    我们将这个问题拆成三个部分:
    \begin{enumerate}
        \item 人际关系重要吗?
        \item 在大学里,上面问题的回答的支撑论据是否依然成立?
        \item 大学毕业会对上面的支撑论据造成什么影响?
    \end{enumerate}
    然后分别判断并作出(符合 Geno 本人的观点,但并不一定可以作为行事指导的)回答。

    \subsection{人际关系重要吗?}\label{subsec:Uncategorized/IsUniversityInterpersonImportant/IsInterpersonImportant}
        是的,在当今社会的大多数场景下,人际关系都是重要的。
        \begin{enumerate}
            \item 当今社会的生产力高度发达,满足人类不同层次的需求所需要的工作的分工非常详细且明确,人与人之间的合作是人自身发展的需要也是社会进步的动力。在这种场景下,完全抛弃人际关系,几乎相当于抛弃了现代社会。
            \item 在当今社会,人类赖以生存所需要的工作远比古代的庞大。在古代,一个人独立生存所需要的技能并不非常多,人类依然有能力掌握相关的技能;但在随着社会发展,在追求生活质量的现代,一个人独立生存所需要的技能已经是一般人无法接受的程度(可以看看贝爷感受一下),绝大多数情况下一个人都需要其他人的工作以更好地享受生活,绝大多数人并不愿意在生活方面当一个苦行僧。
            \item 现代社会的大多数工作都非常庞大,单人所做的工作往往是一个庞大工作分解后的一小块任务。如果没有人际关系的管理的话,每个任务完成的结果就不一定会与大方向一致,从而可能打乱整个工作。
        \end{enumerate}

    \subsection{大学里依然成立吗?}\label{subsec:Uncategorized/IsUniversityInterpersonImportant/IsItInUniversity}
        \begin{enumerate}
            \item 在大学里完全独立行事不依靠同学可行吗?可行,但可能被周围同学异样地看待。
            \item 在大学里可以不需要其他人的工作而不降低生活享受吗?可以,“不被打扰而沉浸在自己的世界中”。
            \item 在大学里单人的工作与大方向不一致可能打乱整个工作吗?可能,但是更多被吐槽的情况是,别人是打乱工作的,自己是被打乱的那个。
        \end{enumerate}
        所以,虽然有句话说,大学是半个社会,但是上面的这些人际关系重要的原因,并不一定在大学里成立。特别是最后一条,存在一个被广泛厌恶的模式:小组作业。

        小组作业,是大学的课程作业的一种常见形式,其理论上的主要形式为,由多人分工合作完成一个任务。但是,对于被俗称“猪队友”的不合作的行为的吐槽从未停止过,主要的不合作方式表现为不按时完成分工内容或以无法复用的方式完成分工内容。

        从现实上说,对于小组作业的队友的管理,其实也是人际交往的一部分——队友的来源往往是同班同学。但是,班级这一群体的聚合方式,大多数时候归根到底是在高考出成绩前后\footnote{新中国成立之初,高考志愿填报是在出成绩之前,本世纪初部分省份开始改为出成绩之后。}填报志愿后录取或调剂的结果,而鉴于许多人在填报志愿时并没有按照自己的真实兴趣填报\footnote{有的是不清楚自己的兴趣点在哪从而以毕业工资或就业热点为方向进行填报,有的是以听从亲戚意见的父母为指导,有的则被父母“屈打成招”,甚至可能是被父母偷改志愿等。}这一不争的事实,“同班同学”这一群体的志向重合度并不是那么大,从而导致“同班”并不“同志”。

        相比之下,社团等完全以兴趣为参与动力的群体聚合方式,其志向重复度会明显比班级高,但是每个人都有自己的思维方式,我们无法确保别人的意志与自己的相同,因此同志是非常稀少的。

        因此,大学期间的人际关系是重要的,但是并不是所有的人际关系都是必须维持的。我们的精力是有限的,需要把社交方面的心思放在有益于自己与对方短期及长远发展的所谓“有效社交”上,放弃掉那些空浪费自己精力却无法使自己获益的所谓“无效社交。”

    \subsection{毕业后依然成立吗?}\label{subsec:Uncategorized/IsUniversityInterpersonImportant/IsItAfterUniversity}
        人各有志。毕业后,曾经的同学会走上不同的人生道路:有的会继续深造,有的会走上工作岗位,有的会回家继承家里的矿山:你我自己将会认识新的同学或者同事,曾经的同学也会认识更多的人。有时,虽然他们平时默默地躺在通讯录中,但逢年过节一声招呼,大家又会回到熟悉的地方把酒\footnote{过量饮酒有害健康。}言欢。

        过往的同窗,在自己的人生中可能成为得力的助手:来到陌生的城市,可能偶然发现曾经的同学就在此处已经打拼出了一番天地;抑或是在寻找更加高薪的岗位的路上,发现曾经自己帮助过的好友已在目标公司有了话语权。

    因此,毕业之后,大学里的人际关系依然可能在某些看似不经意的时间中,成为自己人生的转折点。

    从这个意义上说,Geno 认为,大学里的人际关系,还是比较重要的。
