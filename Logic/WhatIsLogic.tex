\section{什么是逻辑}\label{sec:Logic/WhatIsLogic}
    在我们的日常生活中,我们会发现并观察到从天文地理到飞鸟走兽的各种各样的事物的外观(概念)与性质(行为),例如“云朵”、“天空”等等,这就是对事物的直接认知。

    随着对这些行为的观察和记录的逐渐增多,我们可以总结出一些“规律”,例如“乌云密布,可能快要下雨了”,“天上鱼鳞斑,晒谷不用翻”等等。这些“规律”就是我们大脑对事物的行为的认识与反映。

    当我们将这些规律结合起来,用于推断更多的规律、性质以用于指导我们的行为或知识时,我们需要一套科学的、系统的方法保证推断过程的正确性。在这个试图保证正确性的过程中,逻辑(Logic)就自然而然地产生了。
