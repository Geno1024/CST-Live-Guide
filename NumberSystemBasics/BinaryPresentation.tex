\section{二进制表示}\label{sec:NumberSystemBasics/BinaryPresentation}
    我们需要找到一种可以只用二进制表示各种数,并且直觉上符合基本的四则运算的二进制数的记录方式,才能满足现代的二进制计算机的要求。另外,由于存储空间是有限的,因此我们要考虑的记录方式的长度也是有限的:在计算机的现实中,我们并不存在数学那样的理论上可以无限添加数字长度的简易记录方式(因此,这些记录方式会有一个“能表达的数的范围”的概念,以后我们将称之为“数据范围”。相关的内容将在后续章节中再详细描述,这里暂时忽略)。

    在本节中,为了讨论的方便,我们暂不讨论小数(同样,后续的章节会介绍小数的记录方式),只考虑一个 $8$ 位的存储整数的设备(即,必须而且只能使用 $8$ 位二进制来表示,若超过 $8$ 位则将其截断,只保留最右边的 $8$ 位)。

    我们非常容易想到的记录方式是,直接将要记录的数字转成二进制,比如说将 $1$ 表达为 $0000\ 0001$,将 $2$ 表达为 $0000\ 0010$ 等,但是这种表达方式的弊端也非常明显:它无法记录负数。为了记录负数,我们首先要经过一段探索的历程,以取得一种符合期望的记录方式。

    总的来说,我们希望它:
    \begin{itemize}
        \item 能与现有的二进制记录方式尽可能兼容;
        \item 互为相反数的两个数相加结果为 $0$;
        \item 减去一个数,等于加上这个数的相反数;
        \item 不断“加 1”之后能走遍一个范围内的全部数,并且在溢出截断之后再次加“回”到某个数的时候,其所表达的数值不变。
    \end{itemize}

    我们将我们所期望的记录方式分为两部分,一部分是表达正数的规范,另一部分是表达负数的规范,并分别考虑:
    \begin{itemize}
        \item 首先,借鉴我们现有的把正负号放在数的最开头的写法,我们任意地定义最左边的一位为符号位,并且任意地定义 $0$ 表示正数,$1$ 表示负数,剩下的位用来表达实际的数。
        \item 然后,我们定义,如果表达的是正数,那么“实际的数”这一节直接采用所欲表达的数的二进制形式。
        \item 剩下的就是如何定义负数的表达了。
    \end{itemize}

    利用最后一条规则,我们可以观察从 $1$ 连续减去两个 1 以到达 $-1$ 的过程,以进行如下的思考:

    二进制 $0000\ 0001$ 减去 $0000\ 0001$ 之后得到了 $0000\ 0000$,再 减去 $0000\ 0001$ 时就会发生“不够减”的情况,这便让我们联想到小学时学减法所学到的方式:可以往“前”去“借 1”。

%    首先我们结合负数的表示方式,很容易想到两种表达:
%    \begin{enumerate}
%        \item 将最左边一位用于表示正负,剩下的位表达这个数的绝对值;
%        \item 全部位表达这个数的绝对值,如果是负数则每一位都进行取反。
%    \end{enumerate}
%
%    很容易发现,这两种表达方式,都能满足要求 1。但要求 2 与 3 并不容易满足:
%    \begin{itemize}
%        \item 假设我们有两个数 $-2$ 和 $2$,考虑第一种表达方式则它们将被表达为 $1000\ 0010$ 和 $0000\ 0010$,相加的结果是 $1000\ 0100$ 而非 $0$,不满足要求 2;考虑第二种表达方式则它们将被表达为 $1111\ 1101$ 和 $0000\ 0010$,相加的结果则是 $1111\ 1111$,虽然它确实也是 $0$(按照第二种表达方式的定义,$1111\ 1111$ 是 $0$ 的每一位取反的结果,也就是 $-0$),但是看上去生造出 $0$ 的正负的区别似乎并无必要。
%        \item 并且第二种表达方式不满足要求 3:$2 + (-0)$ 在这个记录方式下会记录为 $0000\ 0010 + 1111\ 1111$,其结果(截断后)会是 $0000\ 0001$ 也即是 $1$,这也是不实用的。
%    \end{itemize}
%
%    观察两种表达方式,我们能发现两种表达方式均事实上将最左边的位用来表示正负(对于第二种表达方式来说,这一点并不容易看出来:似乎)
