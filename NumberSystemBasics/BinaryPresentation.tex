\section{二进制表示}\label{sec:NumberSystemBasics/BinaryPresentation}
    我们需要找到一种可以只用二进制表示各种数,并且直觉上符合基本的四则运算的二进制数的记录方式,才能满足现代的二进制计算机的要求。另外,由于存储空间是有限的,因此我们要考虑的记录方式的长度也是有限的:在计算机的现实中,我们并不存在数学那样的理论上可以无限添加数字长度的简易记录方式(因此,这些记录方式会有一个“能表达的数的范围”的概念,以后我们将称之为“数据范围”。相关的内容将在后续章节中再详细描述,这里暂时忽略)。

    在本节中,为了讨论的方便,我们暂不讨论小数(同样,后续的章节会介绍小数的记录方式),只考虑一个 $8$ 位的存储整数的设备(即,必须而且只能使用 $8$ 位二进制来表示,若超过 $8$ 位则将其截断,只保留最右边的 $8$ 位)。

    我们非常容易想到的记录方式是,直接将要记录的数字转成二进制,比如说将 $1$ 表达为 $0000\ 0001$,将 $2$ 表达为 $0000\ 0010$ 等,但是这种表达方式的弊端也非常明显:它无法记录负数。为了记录负数,我们首先要经过一段探索的历程。
