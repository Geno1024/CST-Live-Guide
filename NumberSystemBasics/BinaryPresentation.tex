\section{二进制表示}\label{sec:NumberSystemBasics/BinaryPresentation}
    我们需要找到一种可以只用二进制表示各种数,并且直觉上符合基本的四则运算的二进制数的记录方式,才能满足现代的二进制计算机的要求。另外,由于存储空间是有限的,因此我们要考虑的记录方式的长度也是有限的:在计算机的现实中,我们并不存在数学那样的理论上可以无限添加数字长度的简易记录方式(因此,这些记录方式会有一个“能表达的数的范围”的概念,以后我们将称之为“数据范围”。相关的内容将在后续章节中再详细描述,这里暂时忽略)。

    在本节中,为了讨论的方便,我们暂不讨论小数(同样,后续的章节会介绍小数的记录方式),只考虑一个 $8$ 位的存储整数的设备(即,必须而且只能使用 $8$ 位二进制来表示,若超过 $8$ 位则将其截断,只保留最右边的 $8$ 位)。

    我们非常容易想到的记录方式是,直接将要记录的数字转成二进制,比如说将 $1$ 表达为 $0000\ 0001$,将 $2$ 表达为 $0000\ 0010$ 等,但是这种表达方式的弊端也非常明显:它无法记录负数。为了记录负数,我们首先要经过一段探索的历程,以取得一种符合期望的记录方式。

    总的来说,我们希望它:
    \begin{itemize}
        \item 能与现有的二进制记录方式尽可能兼容;
        \item 互为相反数的两个数相加结果为 $0$;
        \item 减去一个数,等于加上这个数的相反数。
    \end{itemize}

    首先我们结合负数的表示方式,很容易想到两种表达:
    \begin{enumerate}
        \item 将最左边一位用于表示正负,剩下的位表达这个数的绝对值;
        \item 全部位表达这个数的绝对值,如果是负数则每一位都进行取反。
    \end{enumerate}

    很容易发现,这两种表达方式,都能满足要求 1。但要求 2 并不容易满足: 假设我们有两个数 $-2$ 和 $2$,考虑第一种表达方式则它们将被表达为 $1000\ 0010$ 和 $0000\ 0010$,相加的结果是 $1000\ 0100$ 而非 $0$;考虑第二种表达方式则它们将被表达为 $1111\ 1101$ 和 $0000\ 0010$,相加的结果则是 $1111\ 1111$,都不符合要求。
