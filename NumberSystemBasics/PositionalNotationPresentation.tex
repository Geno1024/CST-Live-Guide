\section{进位制的表示}\label{sec:NumberSystemBasics/PositionalNotationPresentation}
    显然,同一个数在不同的进位制下会有不同的表示,相同表示的数在不同的进位制下也会表示不同的值。因此,我们在讨论不同进位制的时候,需要对进位制做出表示,于是出现了进位制的记号或者说表示方式的问题。
    \subsection{记号表示}\label{subsec:NumberSystemBasics/PositionalNotationPresentation/Natural}
        数学上的各种记号是一个自洽且略显混乱的系统,记号通常由发现或者试图简化某些运算的数学家的想象力、学习的语言数量及该记号所准备表达的含义所决定;而在计算机方面,记号通常会受到输入输出设备及系统的限制。因此,在数学和计算机(编程)方面的记号,会有所不同。
        \begin{itemize}
            \item 数学上较为常见的记号是,将数直接写出,然后将底数作为下标附在其后。例如,二进制的 $1010$ 记为 \[1010_2\]。有时则会将数或基数使用小括号包裹,例如 \[1010_{(2)}\] 或 \[(1010)_2\]。
            \item 程序语言方面,在不同的程序语言与场合中有不同的约定,并且大多数时候仅指定了十进制、十六进制和八进制的记号,少数时候指定了二进制的记号,少见可以指定任意进制的记号。最为常见的记号来自 C 语言,指定了十六进制前缀 \texttt{0x} 和八进制前缀 \texttt{0},另外还有一些例如 \texttt{\#}、\texttt{x} 等十六进制前缀。有时需要根据上下文才能确定具体进制的基底,但是在程序语言方面大多数时候看到的是十进制或十六进制。
        \end{itemize}
