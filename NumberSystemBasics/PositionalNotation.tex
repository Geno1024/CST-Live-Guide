\section{进位制}\label{sec:NumberSystemBasics/PositionalNotation}
    我们回想一下我们婴幼儿时候学习计数的样子:

    伸出一只手的五根手指,弯曲一根手指,再弯曲第二根手指,接着是第三根、第四根、第五根手指,直到整只手变成一个拳头;为了继续计数下去,我们伸出了另一只手,并继续弯曲第六、七、八、九、十根手指。

    \subsection{十进制记数法}\label{subsec:NumberSystemBasics/PositionalNotation/Decimal}
        古人在一开始,也是使用这样的计数方式,这也是沿用至今的最方便的“随身携带”的简单计算工具,成语“屈指可数”也体现了这一点:较为简单的数就是使用“屈指”来计数的。但是随着生产力的发展,“十”已经不能满足要求了,于是祖先们在遇到“十”之后,便在地上放一块小石头,或者在绳子上打一个结,再重新使用双手计数。

        由于进化的偶然,包括人类在内的灵长类动物均具有两只手,每只手上有五根手指。因此,历史上的不少文明不约而同地发展出了“逢十进一”的计数法,这被普遍认为是十进制的生物学起源。

        使用手指来记数,有一些明显的缺点,比如说它无法记录。因此,我们创造出了十个汉字,即一、二、三、四、五、六、七、八、九、十,以记录数字。当要记录的数比“十”大的时候,我们会继续叠加使用这些汉字,例如“十一”、“十二”,乃至“二十”,一直到“九十九”。在“九十九”之后,我们又需要新的汉字,即“百”,来表示十个“十”的数值。对于更大的数,我们还需要用于表达十个“百”的“千”、用于表达十个“千”的“万”,等等。

        \begin{displayquote}
            汉语中对于更大的数字的权重,传统上有亿、兆、京、垓、秭、穰、沟、涧、正、载十个名称。但对于这些名称具体表达的权重大小存在不同的定义,这也造成了现代汉语表达上的混淆。

            \begin{itemize}
                \item 下数:每十倍使用下一个权重名称\footnote{“十十变之”\cite[卷上 12]{wujin},“以十进”\cite[卷一 15]{shulijinyun}},即 10 万为 1 亿($10^6$),10 亿为 1 兆($10^7$),10 兆为 1 京($10^8$),依此类推。
                \item 万进:每万倍使用下一个权重名称\footnote{“以万进”\cite[卷一 15]{shulijinyun}},即 1 万万为 1 亿($10^8$),1 万亿为 1 兆($10^{12}$),1 万兆为 1 京($10^{16}$),依此类推。
                \item 中数:每万万倍使用下一个权重名称\footnote{“万万变之”\cite[卷上 12]{wujin},“万万曰亿,万万亿曰兆,……”\cite[卷上 3]{sunzi}},即 1 万万为 1 亿($10^8$),1 万万亿为 1 兆($10^{16}$),1 万万兆为 1 京($10^{24}$),依此类推。
                \item 上数:递归地使用更小的权重名称,直到用完才使用更大的权重名称\footnote{“数穷则变”\cite[卷上 12]{wujin},“以自乘之数进”\cite[卷上 15]{shulijinyun}},即 1 万万为 1 亿($10^8$),1 亿亿为 1 兆($10^{16}$),1 兆兆为 1 京($10^{32}$),依此类推。
            \end{itemize}
            另外,《中华人民共和国法定计量单位》又定义了“兆”为 $10^6$。\footnote{所表示的因数:$10^6$,词头名称:兆,词头符号:M\cite[附件 表 5]{gf1984-28}}

            因此,“兆”字在不同的上下文中会有不同的所指,需要根据语境及地区以确定其所表示的值。

            另外,佛教中存在着表达更大的数的名称,递进规律使用了“上数”的定义\footnote{“百千百千名一拘梨。拘梨拘梨名一不变。……不可说转不可说转名一不可说转转。”\cite[卷第二十九 1]{huayan}},但这些名称并没有现实用途。
        \end{displayquote}

        而古印度人,创造出了九个符号,演化至今成为了最流行的“阿拉伯数字”,即 \texttt{1}、\texttt{2}、\texttt{3}、\texttt{4}、\texttt{5}、\texttt{6}、\texttt{7}、\texttt{8}、\texttt{9}。当要记录的数字比 \texttt{9} 大的时候,我们往“前”“进一”并使用新符号 \texttt{0} 进行占位。这个“进”导致相同的数码在不同的位置上会表示不同的意义,例如 \[100\] 的 \texttt{1} 表示一个“百”,而 \[10000\]的 \texttt{1} 表示一个“万”。至于为何“进”的方向是向左而非向右,这只是约定俗成的原因。

        “十进制记数法”(Decimal Notation),就是使用十个符号记录数字,并在记录超出“十”的范围的数字时,在不同的位置上使用这十个符号表征不同“权重”的值的记数系统。

    \subsection{通用的进位制记数法}\label{subsec:NumberSystemBasics/PositionalNotation/Common}
        “进位制记数法”(Positional Notation),或者称为“位值制记数法”(Place-Value Notation),就是将十进制的定义推广到十以外的任意的数上,使用若干个不同的符号记录数字,并且使用不同的位置表达同一个符号的不同“权重”的记数系统。

        在进位制中,不同的位置上的数码,要乘以它所在的位的“权重”。对于十进制而言,这个“权重”就是我们通常所说的“个”、“十”、“百”、“千”、“万”等:当记录的数字达到 9 之后,下一个数“逢十进一”,将第二位增添 1,而当前位被清零,整个数变成了 10;同样,在第三位上,要等待第二位被第一位累积到 9 之后,再被第一位的 9 由于逢十进一而导致连进两位从而才增添 1。于是,在最右一位上的数字,代表的就是有多少个单位“1”;而它左边一位的数字,代表的是有多少个“十”,再往左一位的数字,代表有多少个“百”,或者说有多少个“十个十”。由此可以发现,某一位上的“权重”其实就是进位所需要的数的大小,自乘若干次的结果,而这个次数就是它与最右边一位的距离。因此,这个“进位所需要的数”,或者说“逢几进一”的“几”,是一个进位制记数法中的关键值。

        “底数”,或者称为“基数”(Base),定义为一个进位制记数法中的可用数码的数量,同时也是一个“逢几进一”的“几进制记数法”中的“几”。

        在有了通用的进位制的概念之后,我们可以非常容易地定义其它的进位制。

    \subsection{二进制}\label{subsec:NumberSystemBasics/PositionalNotation/Binary}
        二进制(Binary),即以 \texttt{2} 为基数的进位制。现代意义上的以 2 为基数的进位制,是由 Gottfried Wilhelm Leibniz 受伏羲八卦图的影响在 1703 年完成的论文\cite{leibniz-binary}中提出的。
        % TODO: 改写上下两段话。
        由于二进制的每一位只有两种状态,因此使用若干个可以表达两种状态的物体即可表达一个二进制数,而电子电路恰好满足这个条件,例如电路的“通”与“断”,电压的“高”与“低”,等等,因此现代的计算机都以二进制作为电路层面最底层的数学基础。

        根据前文所表达的“进位制”的概念,我们可以很容易地推出二进制中的“逢二进一”是如何应用到整数的表示中的:
        \begin{enumerate}[start = 0]
            \item 首先从 \texttt{0} 开始;
            \item 下一个数为 \texttt{1};
            \item 再下一个数即满足了逢二进一的条件,因而为 \texttt{10};
            \item 下一个数为 \texttt{11};
            \item 再下一个数时在第二位和第一位上均满足了“逢二进一”,因而连续进两位而成为 \texttt{100};
            \item ……
        \end{enumerate}

        可以看出,二进制虽然状态数简单,但是表达出来的长度随着所要表达的数的增大而快速增长,因而二进制的表达通常过于冗长。就此我们需要一些二进制的衍生进制。

    \subsection{二进制的衍生进制}\label{subsec:NumberSystemBasics/PositionalNotation/BinaryDerivation}
        最容易想到也几乎是唯一的二进制衍生方式,就是把一个二进制数进行分组,把连续的几位“合并”到一起,并使用新的数码来表示它们。

        常见的二进制的衍生进制是八进制(Octal)和十六进制(Hexadecimal)。八进制通过把二进制每三位压缩为一位的方式实现合并,而十六进制则是每四位合并为一位。

        顾名思义,八进制逢八进一,需要八个数码。为了复用已有的符号,一般八进制使用十进制所使用的前八个符号,即 \texttt{0} 到 \texttt{7} 八个数码。

        而十六进制,需要十六个数码,通常的十进制并不能满足需要,因此需要额外使用六个符号。在计算机的历史上,有过多种不同的符号选择方式:
        \begin{savenotes}
            \begin{table}
                \centering
                \begin{tabular}{|c|l|l|l|l|l|l|}
                    \hline
                    设备型号 & 10 & 11 & 12 & 13 & 14 & 15 \\ \hline
                    Monrobot XI\footnote{\cite[Sexadecimal System]{monrobot-xi-program-manual},有文章指出这样的奇怪排布是出于穿孔卡片方便\cite{johnmann-monrobot-xi},但是 Geno 并没有找到这个型号的穿孔卡片的样子。} & S & T & U & V & W & X \\ \hline
                    DATAmatic 1000\footnote{“Similarly, "hexadecimal" is used to denote any of the sixteen characters (ten decimal digits and six special symbols called hex B, C, D, E, F, and G)$\cdots$”\cite[Programmer's Language: Constants]{datamatic-1000-manual-vol-1},不使用 A 的原因\textit{可能}是为了避免与下文说明的使用 A 作为 Alphanumeric constant 的声明相混淆。}
                    PDS 1020 & L & C & A & S & M & D \\ \hline
                \end{tabular}
                \caption{历史上出现过的不同的十六进制数码}
                \label{tab:NumberSystemBasics/PositionalNotation/BinaryDerivation/HistoricalHexadecimalDigits}
            \end{table}
        \end{savenotes}
