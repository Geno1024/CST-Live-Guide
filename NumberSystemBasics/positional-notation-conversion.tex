\section{不同进位制的转换}\label{sec:NumberSystemBasics/positional-notation-conversion}
    根据进位制的原理,我们可以计算出一个(我们所熟悉的)十进制数所对应的其他进制的表示,也可以将一个其他进制的表示转换为十进制。这样,借助我们所熟悉的十进制,我们就能做到任意进制的相互转换。

    若源进制与目标进制的基数都是某一个数的整数次幂,或者非常熟悉目标进制的运算的话,我们也可以不借助十进制直接进行不同进位制的转换。

    \subsection{十进制转换为其他进制}\label{subsec:NumberSystemBasics/positional-notation-conversion/from-decimal}
        将一个十进制数转换为其他进制有几种方法。整数部分可以使用短除法或者退位法,小数部分可以使用连乘法或者退位法,然后将两部分合起来即可。
        \subsubsection{整数的短除法}\label{subsubsec:NumberSystemBasics/positional-notation-conversion/from-decimal/integer-short-division}
            首先,将这个十进制数除以目标进制的以十进制表示的基底,记录该除法结果的余数在目标进制下的表示,并将商继续进行这种除法,直到商为 0 为止,然后将记录的余数从最后到最开始的顺序书写,即为该数的目标进制的表示。

            例如,欲将十进制整数 $13579$ 转换为十六进制:

            \begin{enumerate}
                \item 将该十进制整数除以目标进制的以十进制表示的基底(即 $16$),得到商 $848$,余数 $11$ 转换为目标进制的结果为 $B$;
                \item 商 $848$ 继续除以 $16$ 得到商 $53$,余数 $0$(整除);
                \item $53$ 除以 $16$ 得到商 $3$,余数 $5$;
                \item $3$ 除以 $16$ 得到商 $0$,余数 $3$,短除法结束;
                \item 将计算得到的余数从下往上读取一遍,即得到结果 $(350B)_{16}$。
            \end{enumerate}

            % https://tex.stackexchange.com/a/210652/149813
            \begin{figure}
                \centering
                \begin{tabular}{rrlll}
                    16 \shortdiv{13579} &    &                 &   &                        \\
                    16   \shortdiv{848} & 11 & \textrightarrow & B & \tikzmark{sdi-end}     \\
                    16    \shortdiv{53} &  0 & \textrightarrow & 0 &                        \\
                    16     \shortdiv{3} &  5 & \textrightarrow & 5 &                        \\
                    0                   &  3 & \textrightarrow & 3 & \tikzmark{sdi-start}
                \end{tabular}
                \begin{tikzpicture}[overlay, remember picture]
                    \draw[->] ($(pic cs:sdi-start)+(0pt,.5ex)$) to ($(pic cs:sdi-end)+(0pt,.5ex)$);
                \end{tikzpicture}
                \caption{短除法转换十进制整数为其他进制的示例}
                \label{fig:NumberSystemBasics/positional-notation-conversion/from-decimal/integer-short-division/sample}
            \end{figure}
        \subsubsection{整数的退位法}\label{subsubsec:NumberSystemBasics/positional-notation-conversion/from-decimal/integer-descending-subtraction}
            将目标进制的的基底的乘方序列以十进制表示从 $1$ 开始往左写,直到比这个十进制数大为止。然后将这一行数从左向右依次与这个数比较,如果这个十进制数比较大的话,就将这个十进制数减去对应的数若干次直到比对应的数小,并且在其下方写上减去的次数在目标进制下的表示,否则就跳过并在其下方写上 $0$,直到这个数被减到 $0$ 为止,如果这一行数还未结束的话则在剩下的数下方均写上 $0$,以此得到的第二行数即为该数的目标进制的表示。

            例如,欲将十进制整数 $3096$ 转换为八进制:

            \begin{enumerate}
                \item 将目标进制的基底的乘方序列以十进制表示从 $1$ 开始往左写,直到比这个十进制数大为止,得到数列 $4096$、$512$、$64$、$8$、$1$;
                \item 然后将这一行数从左向右依次与这个数比较,第一个是 $4096$,比 $3072$ 大,故在 $4096$ 下方写 $0$;
                \item 下一个数是 $512$,比 $3096$ 小,尝试发现减去 $6$ 次之后的结果 $24$ 开始比 $512$ 小,故在 $512$ 下方写 $6$;
                \item 下一个数 $64$ 比 $24$ 大,故跳过,在 $64$ 下方写 $0$;
                \item 下一个数 $8$ 可以将 $24$ 减去 $3$ 次得到 $0$,故在 $8$ 下方写 $3$,并结束循环;
                \item 最后剩余一个数 $1$,在其下方写上 $0$;
                \item 将第二行数从左向右读取,并舍弃开头的 $0$,即得到结果 $(6030)_8$。
            \end{enumerate}

            \begin{figure}
                \centering
                \begin{tabular}{lrrrrr}
                    乘方序列     & 4096 &  512 & 64 &  8 & 1 \\
                    结果         & 0    &    6 &  0 &  3 & 0 \\ \hline
                    要减去多少   & 0    & 3072 &  0 & 24 &   \\
                    剩余         & 3096 &   24 & 24 &  0 &
                \end{tabular}
                \caption{退位法转换十进制整数为其他进制的示例}
                \label{fig:NumberSystemBasics/positional-notation-conversion/from-decimal/integer-descending-subtraction/sample}
            \end{figure}
        \subsubsection{小数的连乘法}\label{subsubsec:NumberSystemBasics/positional-notation-conversion/from-decimal/fractal-long-multiplication}
            将小数部分乘以目标进制的以十进制表示的基底,记录该乘法结果的整数部分在目标进制下的表示,并将小数部分继续进行这种乘法,直到小数部分为 0 为止,然后将记录的整数部分从最开始到最后的顺序书写,即为该数的目标进制的表示。

            注意,将小数进行进制转换时,很容易出现“死循环”,即永远无法达到“小数部分为 0”的情况。这是非常常见的情况,此时转换的结果会是一个无限循环小数或者无限不循环小数,就像 $\frac{1}{3}$ 在十进制下表示为无限循环小数 $0.\overline{3}$ 而在三进制下简单地表示为 $0.1$ 一样。

            例如,欲将十进制小数 $0.3125$ 转换为二进制:
            \begin{enumerate}
                \item 将小数部分乘以目标禁止的以十进制表示的基底(即 $2$),得到 $0.625$,整数部分为 $0$,小数部分为 $.625$;
                \item 小数部分 $.625$ 继续乘以 $2$ 得到 $1.25$,整数部分为 $1$,小数部分为 $.25$;
                \item $.25$ 乘以 $2$ 得到 $0.5$,整数部分为 $0$,小数部分为 $.5$;
                \item $.5$ 乘以 $2$ 得到 $1$,整数部分为 $1$,小数部分为 $.$,连乘法结束;
                \item 将计算得到的整数部分从上往下读取一遍,即得到结果 $(0.0101)_2$。
            \end{enumerate}

            \begin{figure}
                \centering
                \begin{tabular}{llll}
                    \tikzmark{lmf-start}    &   &          & .3125 \\
                    &   & $\times$ &     2 \\ \hline
                    & 0 &          & .625  \\
                    &   & $\times$ &    2  \\ \hline
                    & 1 &          & .25   \\
                    &   & $\times$ &   2   \\ \hline
                    & 0 &          & .5    \\
                    &   & $\times$ &  2    \\ \hline
                    \tikzmark{lmf-end}      & 1 &          & .
                \end{tabular}
                \begin{tikzpicture}[overlay, remember picture]
                    \draw[->] ($(pic cs:lmf-start)+(0pt,.5ex)$) to ($(pic cs:lmf-end)+(0pt,.5ex)$);
                \end{tikzpicture}
                \caption{连乘法转换十进制小数为其他进制的示例}
                \label{fig:NumberSystemBasics/positional-notation-conversion/from-decimal/fractal-long-multiplication}
            \end{figure}

        \subsubsection{小数的退位法}\label{subsubsec:NumberSystemBasics/positional-notation-conversion/from-decimal/fractal-descending-subtraction}

            与整数的退位法类似,但是所需要的乘方序列的指数是负数的数列。

            例如,欲将十进制小数 $0.57421875$ 转换为十六进制:

            \begin{enumerate}
                \item 将目标进制的基底的负数乘方序列以十进制表示从 $1$ 开始往右写,由于无法事先知道有多少位因此先写一位,得到数列 $1$、$0.0625$;
                \item 然后从 $1$ 的下一个数开始,试着减去足够多次的 $0.0625$ 但不变成负数,将能够减去的最大次数 $9$ 写于其下方,并记录减去之后的剩余 $0.01171875$;
                \item 再计算一位乘方序列也就是 $16^{-2}$,得到 $0.00390625$;
                \item 将刚刚的剩余 $0.01171875$ 做如上的试减,将能够减去的最大次数 $3$ 写于其下方,减去之后的剩余为 $0$,故结束循环;
                \item 将第二行数从左向右读取,并补上开头的 $0.$,即得到结果 $(0.93)_{16}$。
            \end{enumerate}

            \begin{figure}
                \centering
                \begin{tabular}{llll}
                    乘方序列   & 1          & 0.0625     & 0.00390625 \\
                    结果       & 0          &          9 &          3 \\ \hline
                    要减去多少 & 0          & 0.5625     & 0.01171875 \\
                    剩余       & 0.57421875 & 0.01171875 & 0
                \end{tabular}
                \caption{退位法转换十进制为其他进制的示例}
                \label{fig:NumberSystemBasics/positional-notation-conversion/from-decimal/fractal-descending-subtraction}
            \end{figure}
    \subsection{其他进制转换为十进制}\label{subsec:NumberSystemBasics/positional-notation-conversion/to-decimal}
        将一个非十进制的数转换为十进制,只需要将各位上的数与对应位的权重相乘并求和即可。

        例如,欲将十六进制数 $(BAD.BEEF)_{16}$ 转换为十进制:
        \begin{enumerate}
            \item 将原数的每一位都转为十进制;
            \item 从小数点开始,往上是源基底的 $0$ 次方并将指数递增,往下是源基底的 $-1$ 次方并将指数递减,写出每一位的权重;
            \item 每一位均乘以对应的权重,得到积;
            \item 将积求和,即得到结果 $2989.7458343505859375$。
        \end{enumerate}

        \begin{figure}
            \centering
            \begin{tabular}{r|lrrlrl}
                原数 &             十进制 &          &      权重 &   &  积 \\
                B    & \textrightarrow 11 & $\times$ & $16^2$    & = & 2816 &. \\
                A    & \textrightarrow 10 & $\times$ & $16^1$    & = &  160 &. \\
                D    & \textrightarrow 13 & $\times$ & $16^0$    & = &   13 &. \\
                .    & \textrightarrow  . & $\times$ &  .        & = &      &. \\
                B    & \textrightarrow 11 & $\times$ & $16^{-1}$ & = &    0 &.6875 \\
                E    & \textrightarrow 14 & $\times$ & $16^{-2}$ & = &    0 &.0546875 \\
                E    & \textrightarrow 14 & $\times$ & $16^{-3}$ & = &    0 &.00341796875 \\
                F    & \textrightarrow 14 & $\times$ & $16^{-4}$ & = &    0 &.0002288818359375 \\
                &                     &          &           & + &                      \\ \hline
                &                     &          &           &   & 2989 &.7458343505859375
            \end{tabular}
            \caption{转换十六进制数为十进制数的示例}
            \label{fig:NumberSystemBasics/positional-notation-conversion/to-decimal/positional}
        \end{figure}
    \subsection{存在乘方关系的不同进制的简便转换}\label{subsec:NumberSystemBasics/positional-notation-conversion/powered-base}
        如果源进制和目标进制都是同一个数的整数次幂,那么可以简单地把原数一位一位地转化为该数,然后以目标进制的指数作为一组的数量重新分组,再把每组都转化为一个目标禁止的数码,即可不经过十进制进行直接的转换。

        例如,欲将二十七进制数 $(GENO.HI)_{27}$ 转化为九进制 :
        \begin{enumerate}
            \item 检查源进制与目标进制,发现源进制 $27 = 3 ^ 3$,目标进制 $9 = 3 ^ 2$,均为 $3$ 的整数次幂,符合这个简便转换的条件;
            \item 将源进制逐位转化为中间表示,即 $3$ 进制的数 $(121\ 112\ 212\ 220.122\ 200)_3$;
            \item 将中间表示按照目标进制的指数作为一组,从小数点开始向两边重新分组,得到 $(12\ 11\ 12\ 21\ 22\ 20.12\ 22\ 00)_3$;
            \item 将每一组分别转为目标进制,即得结果 $(545786.48)_9$,整数部分最高位和小数部分最低位的 $0$ 可以省略。
        \end{enumerate}

        \begin{figure}
            \centering
            \begin{tabular}{lrrrrrrr}
                源数     &   G &   E &   N &   O & . &   H &   I \\
                中间表示 & 121 & 112 & 212 & 220 & . & 122 & 200 \\ \hline
            \end{tabular}
            \begin{tabular}{lrrrrrrrrrr}
                重新分组 & 12 & 11 & 12 & 21 & 22 & 20 & . & 12 & 22 & 00 \\
                结果     &  5 &  4 &  5 &  7 &  8 &  6 & . &  4  &  8 &  0
            \end{tabular}
            \caption{相同整数的不同幂次进制之间简便转换的示例}
            \label{fig:NumberSystemBasics/positional-notation-conversion/powered-base/with-intermediate}
        \end{figure}
