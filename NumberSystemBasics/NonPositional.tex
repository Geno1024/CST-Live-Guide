\section{非进位制的记数系统}\label{sec:NumberSystemBasics/NonPositional}
    我们前面讨论的都是进位制的记数系统,但除此之外还有一些非进位制的记数系统(Non-positional Number System)。

    非进位制的记数系统,是数码所表示的数值与位置无关的记数系统的统称。所谓“数码所表示的数值与位置无关”的含义,可以先回想一下前面在描述进位制时的内容:

    \begin{displayquote}
        “进位制记数法”……就是……使用若干个不同的符号记录数字,并且使用不同的位置表达同一个符号的不同“权重”的记数系统。

        在进位制中,不同的位置上的数码,要乘以它所在的位的“权重”。对于十进制而言,这个“权重”就是我们通常所说的“个”、“十”、“百”、“千”、“万”等:……于是,在最右一位上的数字,代表的就是有多少个单位“1”;而它左边一位的数字,代表的是有多少个“十”,再往左一位的数字,代表有多少个“百”,或者说有多少个“十个十”。
    \end{displayquote}

    这就是“数码所表示的数值与位置有关”的一个例子:例如对于十进制数 $11000$ 而言,第一个 $1$ 表示的是

    \subsection{二进制编码的十进制}\label{subsec:NumberSystemBasics/NonPositional/BinaryCodedDecimal}
        二进制编码的十进制(Binary-Coded Decimal,简称 BCD)是
