\section{非进位制的记数系统}\label{sec:NumberSystemBasics/NonPositional}
    我们前面讨论的都是进位制的记数系统,但除此之外还有一些非进位制的记数系统(Non-positional Number System)。

    非进位制的记数系统,是数码所表示的数值与位置无关的记数系统的统称。所谓“数码所表示的数值与位置无关”的含义,可以先回想一下前面在描述进位制时的内容:

    \begin{displayquote}
        “进位制记数法”……就是……使用若干个不同的符号记录数字,并且使用不同的位置表达同一个符号的不同“权重”的记数系统。

        在进位制中,不同的位置上的数码,要乘以它所在的位的“权重”。对于十进制而言,这个“权重”就是我们通常所说的“个”、“十”、“百”、“千”、“万”等:……于是,在最右一位上的数字,代表的就是有多少个单位“1”;而它左边一位的数字,代表的是有多少个“十”,再往左一位的数字,代表有多少个“百”,或者说有多少个“十个十”。
    \end{displayquote}

    这就是“数码所表示的数值与位置有关”的一个例子:例如对于十进制数 $11000$ 而言,第一个 $1$ 表示的是“一个万”,第二个 $1$ 表示的是“一个千”,相同的数码在不同的位置上表示不同的含义,且位置信息在数值表记中作为权重体现,这就是“数码所表示的数值与位置有关”的情况。

    “数码所表示的数值与位置无关”的情况则与此相反:一个数码在某个位置上,并不表示它必然具有某个指定的权重,它所代表的数值可能要根据其上下文确定。

    \subsection{罗马数字}\label{subsec:NumberSystemBasics/NonPositional/RomanNumeral}
        罗马数字(Roman Numeral)是古罗马使用的记数系统,使用拉丁字母的一套结合规则表达数字,现今依然见于例如年份记载、钟表、编号等场合。

        具体的值见于表~\ref{tab:NumberSystemBasics/NonPositional/RomanNumeral/Values}。

        \begin{table}
            \centering
            \begin{tabular}{|l|c|c|c|c|c|c|c|}
                \hline
                字母 & I & V &  X &  L &   C &   D &    M \\ \hline
                数值 & 1 & 5 & 10 & 50 & 100 & 500 & 1000 \\ \hline
            \end{tabular}
            \caption{罗马数字常用的拉丁字母及其值}
            \label{tab:NumberSystemBasics/NonPositional/RomanNumeral/Values}
        \end{table}

        确定的结合规则包括:
        \begin{enumerate}
            \item 连续重复几次就是几倍;
            \item 数值大的字母的右边写上数值小的字母,是两个数值相加;若是左边则是相减。
        \end{enumerate}

        另外有一些常见的规则,例如连续重复的次数不超过 3 次等,但是并不是严格的;还有一些不常用的表示其他数值的字母,此处不再赘述。

    \subsection{格雷码}\label{subsec:NumberSystemBasics/NonPositional/GrayCode}
        格雷码(Gray Code)是 1953 年由美国工程师 Frank Gray 提出的一种令相邻的数字只相差一位的二进制记数方式,常用于数据传输等场合。

        自然数转换为格雷码之后的序列可以以如下方式构造:
        \begin{itemize}
            \item 令 $a_0$ 为 ${0, 1}$;
            \item 令 $a_n$ 为 $a_{n-1}$ 与逆序且每一个数的最左边补上一个 $1$ 的 $a_{n-1}$;
            \item 则 $a_n$ 为 $0$ 到 $2^{n-1}$ 之间的自然数的格雷码表示。
        \end{itemize}

        二进制数转换为格雷码,只需要将二进制数的相邻两位做不进位加法然后开头补上一个 $1$ 即可;格雷码转换为二进制数则需要先写上开头的 $1$,然后从左起第二位格雷码开始,将每一位格雷码与上一位二进制位做不进位加法得到下一位二进制位。

    \subsection{二进制编码的十进制}\label{subsec:NumberSystemBasics/NonPositional/BinaryCodedDecimal}
        二进制编码的十进制(Binary-Coded Decimal,简称 BCD)是 1928 年由 IBM 发明的一种将十进制数逐位转化为二进制来拼接的记数方式,在一些不需要运算的场合(例如屏幕数字显示等)比较常用。

        最常用的 BCD 是 8421 BCD,即每一位十进制数都转化为 $4$ 位二进制数,而这 $4$ 位的权重从左到右分别是 $8$、$4$、$2$、$1$。此外其它场合常用的还有 5421 BCD、4221 BCD 等。

        另有一种称为“余 3 码”(Excess 3 Code)的 BCD,其构造方式是在转化为 8421 BCD 的时候,在每一位都加上 $3$ 之后再转化为二进制。这种方式的好处是在进行加减法运算时较为方便,并且每一位十进制转化后不存在 $0000$ 和 $1111$,在数据传输时可以简单地排除一些错误。

        除此之外还有使用格雷码来作为二进制编码方式的 BCD。
