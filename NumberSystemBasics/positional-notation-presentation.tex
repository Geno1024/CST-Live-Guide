\section{进位制的表示}\label{sec:NumberSystemBasics/positional-notation-presentation}
    同一个数在不同的进位制中会有不同的表现形式,于是出现了不同进位制之间的表示与转换的问题。
    \subsection{记号表示}\label{subsec:NumberSystemBasics/positional-notation-presentation/natural}
        在数学上,记号通常由发明记号的数学家的想象力与其所学习的语言数量决定;而在计算机相关的方面,记号通常会受到输入输出的设备与系统的限制,因此数学和计算机(编程)方面的表示会有所不同。
        \begin{itemize}
            \item 数学上较为常见的记号是,将数直接写出,然后将底数作为下标附在其后。例如,二进制的 $1010$ 记为\[1010_2\]。有的记号则将数或基数使用小括号包裹,例如\[1010_{(2)}\]或\[(1010)_2\],这些都是常见的数学上的进位制记号。
            \item 程序语言方面,在不同的场合与不同的程序语言中有不同的记号,并且大多数仅指定了记录十进制、十六进制和八进制的记号,少数指定了二进制的记号,少见可以指定任意进制的记号。最为常见的记号来自 C 语言,指定了十六进制前缀 \texttt{0x} 和八进制前缀 \texttt{0},另外还有见于颜色表示的十六进制前缀 \texttt{\#},见于 HTML 编码的 \texttt{x} 等。有时需要根据上下文才能判断具体进制的基底。
        \end{itemize}
    \subsection{二进制表示}\label{subsec:NumberSystemBasics/positional-notation-presentation/binary}
        由于现代计算机均采用二进制,因此需要找到一种可以只用二进制表示各种数,并且符合基本运算法则的方式。另外,由于存储空间的限制,我们需要考虑一个有限长度的记录方式,而非像数学那样理论上可以无限添加长度的记录方式(因此,这些记录方式通常存在一个能表达的数的范围,称为“数值范围”。关于数值范围,以及超出数值范围的数的表达,将在后续章节中再描述,此处暂时忽略)。

        为了讨论方便和章节简短,在本小节中,我们暂不讨论小数(同样,后续章节会介绍小数的记录方式),只考虑一个 $8$ 位的存储整数的设备(即,必须而且只能使用 $8$ 位二进制来表示,若超过 $8$ 位则将其截断,只保留最右边的 $8$ 位)。

        最容易想到的记录方式为,直接记录将源数字转换为二进制之后的结果,比如说将 $1$ 表达为 $0000\ 0001$,把 $2$ 表达为 $0000\ 0010$ 等,但是此种方式无法记录负数,因此创造了几种记录方式以便记录负数并令负数可计算。
        % TODO: 补充历史
        \subsubsection{补数}\label{subsubsec:NumberSystemBasics/positional-notation-presentation/binary/complement}
            对于一个 k 位 N 进制数 a,我们定义 a 的基补数(Radix Complement)为 $N ^ k - a$;定义 a 的减基补数(Diminished Radix Complement)为 $(N ^ k - 1) - a$。

            特别地,在我们当前讨论的上下文中,$k = 8$ 且 $N = 2$,则我们有:

            数 a 的二进制的二补数(2's Complement)为 $2 ^ 8 - a$,一补数(1's Complement)为 $(2 ^ 8 - 1) - a$。

            一个更加直观的计算方式是,对于一个正数的相反数,有:
            \begin{itemize}
                \item 一补数为该数二进制表达的各位取反(即,$1$ 变为 $0$,$0$ 变为 $1$);
                \item 二补数为该数二进制表达的各位取反之后加 $1$ 的结果。
            \end{itemize}

            使用补数可以将减法变成加法,于是使用加法器既可以计算加法也可以计算减法,从而简化运算电路。例如,计算 $2 - 3$,可以转化为 $2 + (-3)$,然后使用一补数计算:
            \[0000\ 0010 + 1111\ 1100 = 1111\ 1110\]
            ,即可得到结果 $-1$;或者使用二补数计算:
            \[0000\ 0010 + 1111\ 1101 = 1111\ 1111\]
            ,也能得到正确结果 $-1$。

            注意,使用一补数计算时,如果发生了溢出,那么需要将溢出的值移动到最低位并相加,才能得到正确的结果。例如计算 $(-3) + (-5)$:
            \[1111\ 1100 + 1111\ 1010 = 1\ 1111\ 0110\]
            ,此时发生了溢出,按照定义,需要将溢出的 $1$ 再加回右边的 $1111\ 0110$ 上,得到 $1111\ 0111$,才能得到正确结果 $-8$。但二补数没有这个问题。

        \subsubsection[原码、反码与补码]{原码、反码与补码\footnote{这似乎是国内特有的概念。Geno 在进行了一定的探索后得出了这个结论,但暂时无法确定该结论是否正确。一补数、二补数的概念似乎只是反码、补码概念的子集:简体中文的文献经常提及“正数的反码、补码……,负数的反码、补码……”,但英文文献似乎仅提及“正数的相反数可以用二补数表示为……”,即简体中文文献将反码与补码视为一种独立的包含了正数、负数与 $0$ 的记数方式,而英文文献仅把补数视为依赖于正数与 $0$ 的一种非正数的记数方式。}}\label{subsubsec:NumberSystemBasics/positional-notation-presentation/binary/chinese-only}
            原码指定,
            \begin{itemize}
                \item 最高位为符号位,并任意地指定 $0$ 表示正数,$1$ 表示负数;
                \item 剩下的位为数值位,记录准备表达的数的绝对值。
            \end{itemize}

            反码指定,当表达正数时,使用数的二进制表达;当表达负数时,使用一补数,即
            \begin{itemize}
                \item 最高位为符号位,$0$ 表示正数,$1$ 表示负数;
                \item 剩下的位为数值位,当表示正数时直接记录准备表达的数的二进制,当表示负数时记录准备表达的数的二进制的各位取反。
            \end{itemize}

            补码指定,当表达正数时,使用数的二进制表达;当表达负数时,使用二补数,即
            \begin{itemize}
                \item 最高位为符号位,$0$ 表示正数,$1$ 表示负数;
                \item 剩下的位为数值位,当表示正数时直接记录准备表达的数的二进制,当表示负数时记录准备表达的数的二进制的各位取反加 $1$ 的结果。
            \end{itemize}

            由于使用补码表达的二进制数,可以在包括正数、负数和 0 范围内的整数中进行正确的加减法,因此现代计算机中均使用补码作为整数的记录方式。
