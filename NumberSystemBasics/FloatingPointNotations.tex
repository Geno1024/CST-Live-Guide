\subsection{浮点数的记录方式}\label{subsec:NumberSystemBasics/FloatingPointNotations}
    在\sout{漫漫的}计算机历史长河中,曾经出现过大量的浮点数记录方式。在被标准化之前,这些浮点数记录方式可以大致分成三类\cite{jjgsavard-2005-cp0201}:
    \begin{enumerate}
        \item $1$ 位符号位,若干位(加上定值以便于记录负数的)指数位,若干位尾数位;
        \item 若干位(二补数记录的)指数位,若干位(二补数记录的)尾数位;
        \item $1$ 位尾数符号位,若干位尾数位,$1$ 位指数符号位,若干位指数位。
    \end{enumerate}
    出现浮点数记录方式百花齐放的原因是,一方面当时也是一个各种计算机百花齐放的年代,另一方面当时的硬件并不非常发达,芯片制造商倾向于复用整数运算模块,而一些特殊的浮点数记录方式方便了在自家芯片上的复用。
