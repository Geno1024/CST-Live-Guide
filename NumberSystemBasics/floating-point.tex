\section{浮点数}\label{sec:NumberSystemBasics/floating-point}
    解决了整数的记录与计算问题之后,我们需要考虑小数的记录与计算的问题。

    我们假定讨论一个 $k$ 位的设备。

    我们可以参考之前学到的方式,直接记录小数的二进制。但是此时会出现一个问题,即数据范围。由于二进制中可用的数码 $0$ 和 $1$ 已经被全部用于表达数,因此已经没有记号用于在数码间标记小数点位置并且不造成歧义,所以需要人为预先规定小数点的位置。假设左起第一位为符号位,然后假设把小数点定在 $n$ 位与 $n + 1$ 位之间,剩下的为小数部分的话,那么可以表达的数的范围为 $(-\underbrace{11 \cdots 11}_{n-1}.\underbrace{11 \cdots 11}_{k-n})_2$ -- $(-0.\underbrace{00 \cdots 00}_{k-n}1)_2$ 和 $(0.\underbrace{00 \cdots 00}_{k-n}1)_2$ -- $(\underbrace{11 \cdots 11}_{n-1}.\underbrace{11 \cdots 11}_{k-n})_2$,分度值为 $(0.\underbrace{00 \cdots 00}_{k-n}1)_2$,或者转为十进制是 $-\frac{2^{(k-1)}-1}{2^{(k-n)}}$ -- $-\frac{1}{2^{(k-n)}}$ 和 $\frac{1}{2^{(k-n)}}$ -- $\frac{2^{(k-1)}-1}{2^{(k-n)}}$,分度值为 $\frac{1}{2^{(k-n)}}$。

    一个具体的例子如\ref{fig:NumberSystemBasics/floating-point/data-range}所示。

    % https://tex.stackexchange.com/a/130008/149813
    \begin{figure}
        \centering
        \begin{tabular}{lS[table-format=-3.7]S[table-format=-3.7]S[table-format=-3.7]S[table-format=-3.7]S[table-format=-3.7]}
            $n$ & 最小值     & 最大值    & 分度值    \\ \hline
            1   & -0.9921875 & 0.9921875  & 0.0078125 \\
            2   & -1.984375  & 1.984375   & 0.015625  \\
            3   & -3.96875   & 3.96875    & 0.03125   \\
            4   & -7.9375    & 7.9375     & 0.0625    \\
            5   & -15.875    & 15.875     & 0.125     \\
            6   & -31.75     & 31.75      & 0.25      \\
            7   & -63.5      & 63.5       & 0.5       \\
            8   & -127       & 127        & 1         \\
        \end{tabular}
        \caption{$k = 8$ 时 $n$ 的值对应的数据范围}
        \label{fig:NumberSystemBasics/floating-point/data-range}
    \end{figure}

    可见,这种记录方式有一个问题,就是无法在精度和范围中间达到平衡:如果采用更小的 $n$,那么可以提高精度,但是可以表达的数的范围便急剧减小;而如果试图提高范围,那么就得采用更大的 $n$,此时精度便开始降低。

    我们仔细思考一下这个平衡失败的原因,就能得出结论:我们没有让“数码在数字中的位置”这个信息变得灵活。如果我们将存储所使用的权重在对数数轴上标记的话,那么会是如图所示的情况:

    % TODO

    为了
