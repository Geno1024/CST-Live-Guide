\section{运算的基础概念}\label{sec:Calculations/OperationBasics}
    \subsection{运算的定义}\label{subsec:Calculations/OperationBasics/DefinitionOfOperation}
        运算(Operation)是对若干个数进行转化以获得结果的操作。在不同的场合下,它可能叫做函数(Function)、映射(Mapping)、算子(Operator)、变换(Transformation)、投影(Projection)等,这些都是不同的领域中对一组数进行操作的过程的称呼。

        对于运算来说,被执行操作的数称为运算数(Operand),但若是对于函数,则将其称为变量(Argument),对于映射则称为原像(Source)。

        参与运算的运算数的数目称为运算的元数(Arity)。例如,没有运算数的运算称为空元运算(Nullary Operation),只有一个运算数的运算称为一元运算(Unary Operation 或 Monadic Operation)或单目运算,有两个运算数的运算称为二元运算(Binary Operation 或 Dyadic Operation)或双目运算,有三个运算数的运算称为三元运算(Ternary Operation 或 Triadic Operation)或三目运算。

        表记一个运算的符号称为运算符(Operator)。另外一个运算是几元的就可以将其运算符称为几元运算符或几目运算符。

    \subsection{运算的表示}\label{subsec:Calculations/OperationBasics/PresentationOfOperation}
        在不同的场合下,运算有不同的表示方式。就运算符的放置位置而言,有前缀(Prefix)、中缀(Infix)、后缀(Postfix)三种方式。

        \begin{itemize}
            \item 前缀表示法,又称波兰表示法(Polish Notation),指的是将运算符放置在运算数之前的表示法。例如将 $1$ 与 $23$ 相加($+$)会记作\texttt{+ 1 23};
            \item 中缀表示法,是我们最经常接触到的表示法,将运算符放置在运算数之间。例如将 $1$ 与 $23$ 相加($+$)会记作\texttt{1 + 23};
            \item 后缀表示法,又称逆波兰表示法(Reverse Polish Notation),则将运算符放置在运算数之后。例如将 $1$ 与 $23$ 相加($+$)会记作\texttt{1 23 +}。
        \end{itemize}

        中缀表示法是我们最熟悉最常用的表示法,我们从小时候学习四则运算开始就已经使用了中缀表示法。但在表示运算的时候存在需要使用括号指定运算优先级的情况,例如对于式 \[1 + 2 \times 3\] 而言,存在先计算 $+$ 还是先计算 $\times$ 的问题,此时需要使用括号进行运算符优先级的指定。

        1924 年,波兰逻辑学家 Jan {\L}ukasiewicz 提出了一种不需要使用括号的新的运算表示法\footnote{... Should we write down the sum of two numbers "$a+b$" as "$+ab$",...\cite{lukasiewicz-1929}},将运算符放在运算数之前。例如对于式 $1 + 2 \times 3$ 而言,
        \begin{itemize}
            \item 若要求先计算 $+$(亦即 $(1 + 2) \times 3$),则后计算的是 $\times$,于是先写上运算符 $\times$,然后写上这个运算符对应的两个运算数 $1\ +\ 2$ 与 $3$,而 $1 + 2$ 使用这种表示法会变成 $+\ 1\ 2$,因此整个式子将表示为 $\times\ +\ 1\ 2\ 3$;
            \item 若要求先计算 $\times$(亦即 $1 + (2 \times 3)$),则后计算的是 $+$,于是先写上运算符 $+$,然后写上这个运算符对应的两个运算数 $1$ 与 $2 \times 3$,而 $2 \times 3$ 使用这种表示法会变成 $\times\ 2\ 3$,因此整个式子将表示为 $+\ 1\ \times\ 2\ 3$。
        \end{itemize}
        这种表示方式写出的式子,例如 $\times\ +\ 1\ 2\ 3$可以以这种方式解读并运算:
        \begin{enumerate}
            \item 首先读到 $\times$,说明要对接下来读到的两个数做一个乘法:$\times\ t_1\ t_2$;
            \item 接着读到 $+$,说明乘法的第一个运算数是由接下来读到的两个数做加法得到的结果得出的:$\times\ +\ a_1\ a_2\ t_2$,上文的 $t_1$ 展开为 $+\ a_1\ a_2$;
            \item 接着依次读到 $1$ 和 $2$,即上面的 $a_1$ 和 $a_2$ 分别为 $1$ 和 $2$,即 $+$ 运算的结果为 $3$:$\times\ +\ 1\ 2\ t_2 \Rightarrow \times\ 3\ t_2$
            \item 接着读到 $3$,说明上文中乘法的第二个运算数 $t_2$ 是 $3$:$\times\ 3\ 3 = 6$。
        \end{enumerate}

        由于前缀表示法能无歧义地表达一个式子且不使用括号,因此前缀表示法被加以使用并推广,称为波兰表示法(Polish Notation)或 {\L}ukasiewicz 表示法。

        不严格地说,可以把诸如 $f(x)$ 这样的记号称为前缀表示法。

        在此之后,将运算符放置在运算数之后的表示法也被发明了出来,它被称为逆波兰表示法(Reverse Polish Notation,RPN),并由于其便于实现的特性而在计算机上广为运用。
