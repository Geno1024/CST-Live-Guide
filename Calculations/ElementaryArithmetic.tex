\section{四则运算}\label{sec:Calculations/ElementaryArithmetic}
    四则运算是我们从小学甚至幼儿园就已经学过的内容。

    \subsection{加法与减法}\label{subsec:Calculations/ElementaryArithmetic/AdditionAndSubtraction}
        加法是将两个量组合起来的操作。

        减法是从一个量中移除另一个量的操作。

    \subsection{乘法与除法}\label{subsec:Calculations/ElementaryArithmetic/MultiplicationAndDivision}

    \subsection{运算及符号的历史}\label{subsec:Calculations/ElementaryArithmetic/HistoryOfSymbol}
        虽然在数千年前人类就已经掌握了加减法运算的操作,但加法符号 $+$ 与减法符号 $-$ 的历史却只有几百年。

        \begin{displayquote}
            1360 年左右 Nicole Oresme 的著作《Algorismus proportionum》使用拉丁语的 $et$ 作为加法符号,据信一份 1417 年的手稿将 $et$ 缩写成现今所熟知的符号,虽然其竖笔并非那么直。1489 年 Johannes Widman 发表的《WebBehende und hübsche Rechnung auff allen Kauffmanschafften》中使用了 $+$ 和 $-$ 符号以表示过剩与不足。1514 年 Giel Vander Hoecke 发表的《Een sonderlinghe boeck in dye edel conste Arithmetica》将 $+$ 与 $-$ 用于现今的加减法操作。
        \end{displayquote}
