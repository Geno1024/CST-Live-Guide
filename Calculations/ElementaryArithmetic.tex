\section{四则运算}\label{sec:Calculations/ElementaryArithmetic}
    四则运算是我们从小学甚至幼儿园就已经学过的内容。

    \subsection{加法与减法}\label{subsec:Calculations/ElementaryArithmetic/AdditionAndSubtraction}
        加法(Addition)是将两个数合并起来的操作。被用于合并的数称为被加数(Augend)\footnote{虽然,由于加法交换律的存在,该称呼已较少使用,一律称为加数。},合并的量称为加数(Addend),合并的结果称为和(Sum)。

        减法(Subtraction)是从一个数中移除另一个数的操作,是加法的逆运算。被用于移除的数称为被减数(Minuend),移除的数称为减数(Subtrahead),移除的结果称为差(Difference)

    \subsection{乘法与除法}\label{subsec:Calculations/ElementaryArithmetic/MultiplicationAndDivision}
        乘法(Multiplication)是加法的连续操作。被用于乘法的操作数称为乘数,乘法的结果称为积(Product)。

        除法(Division)是乘法的逆运算。被用来进行除法的数称为被除数(Dividend),每次除掉的量称为除数(Divisor),结果称为商(Quotient)。如果商被限制为整数,那么未除尽的部分称为余数(Remainder)。

    \subsection{运算及符号的历史}\label{subsec:Calculations/ElementaryArithmetic/HistoryOfSymbol}
        虽然在数千年前人类就已经掌握了加减法运算的操作,但加法符号 $+$ 与减法符号 $-$ 的历史却只有六七百年。

        \begin{displayquote}
            1360 年左右 Nicole Oresme 的著作《Algorismus proportionum》使用拉丁语的 $et$ 作为加法符号,据信一份 1417 年的手稿将 $et$ 缩写成现今所熟知的符号,虽然其竖笔并非那么直。1489 年 Johannes Widman 发表的《WebBehende und hübsche Rechnung auff allen Kauffmanschafften》中使用了 $+$ 和 $-$ 符号以表示过剩与不足。1514 年 Giel Vander Hoecke 发表的《Een sonderlinghe boeck in dye edel conste Arithmetica》将 $+$ 与 $-$ 用于现今的加减法操作。
        \end{displayquote}

        至于乘法符号 $\times$ 与除法符号 $\div$ 的历史则更短。

        \begin{displayquote}
            1628 年 William Oughtred 成书的《Clavis Mathematicae》使用了 $\times$ 作为乘法的符号。1631 年 Thomah Harriot 的书《Analyticae Praxis ad Aequationes Algebraicas Resolvendas》 使用点号作为乘法符号。1659 年 Johann Rahn 的书《Teutsche Algebra》使用星号 $*$ 作为乘法符号,使用 $\div$ 作为除法符号。
        \end{displayquote}

        % 这一节写不下去了。试图配两个图走人吧。

        % 有空补点长除法内容。
