\section{布尔代数}\label{sec:Calcuations/BooleanAlgebra}
    \subsection{起源及建构}\label{subsec:Calculations/BooleanAlgebra/OriginAndBuilding}
        1847 年 10 月 29 日,英国数学家 George Boole 提出了一种描述逻辑关系的数学符号体系,并基于这一套体系对逻辑进行数学分析\cite{boole-1847}。

        首先做出如下的定义:
        \begin{enumerate}
            \item 定义 $1$ 为“全集”\footnote{原文为 Universe。},或者将其理解为对一类不管是否真实存在的物体的想象。另外,一个物体可能不止存在于一个“类”中,因为它可能与其他物体具有不止一种的相同的属性;
            \item 定义例如 $X$、$Y$、$Z$ 等大写字母为一“类”物体。例如,$X$ 代表着一类具有相同属性的物体,$Y$ 代表着一类具有另外一种相同属性的物体;
            \item 定义例如 $x$、$y$、$z$ 等小写字母为对应的大写字母“类”中的相同的属性。例如,$x$ 代表全部 $X$ 的相同的属性,$y$ 代表全部 $Y$ 的相同的属性。
        \end{enumerate}

        在未提及任何具体属性的时候,对于 $X$,我们可以假定 $1$ 就是 $x$,于是我们可以写出式子\[x = x\]来表示“任何具有属性 $x$ 的物体均具有属性 $x$”。

        我们定义两个属性的乘积 $xy$ 代表 $Y$ 中那些能够在 $X$ 中找到的物体,即同时存在于 $X$ 和 $Y$ 中的物体;同理 $xyz$ 代表同时存在于 $X$、$Y$、$Z$ 中的物体:其实这是在描述一个选择\footnote{原文为 elective。}的过程。

        接着我们能得出如下的结论:
        \begin{enumerate}
            \item 选择的结果与选择的分组无关。或者说,对于一类物体,在其中选出属于 $X$ 的,或者将这一类物体任意地分成两份,并分别选出属于 $X$ 的,再将其合并,其结果相同。这可以记为\[x(u + v) = xu + xv\]
            \item 选择的结果与选择的次序无关。例如,我们在全部动物中先选择一群羊,然后在其中选择出有角的\footnote{原文如此。},或者在全部动物中先选择出全部有角的,然后在其中选择出属于羊的,其结果都是选择出了有角的羊\footnote{并不是所有的羊都有羊角,例如承德无角山羊。}。这可以记为\[xy = yx\]
            \item 相同选择执行多次等同于执行一次。在 $X$ 中选出属于 $X$ 的,其结果依然是 $X$。这可以记为\[x ^ n = x\]
        \end{enumerate}

        于是我们建立了一套如下的运算系统:
        \begin{align}
            x(u + v) &= xu + xv \label{eqn:Calculations/BooleanAlgebra/OriginAndBuilding/DistributiveLaw} \\
            xy &= yx \label{eqn:Calculations/BooleanAlgebra/OriginAndBuilding/CommutativeLaw} \\
            x ^ n &= x \label{eqn:Calculations/BooleanAlgebra/OriginAndBuilding/IndexLaw}
        \end{align}

        就此可以构建四个逻辑命题:
        \begin{enumerate}
            \item 全称肯定(Universal Affirmative,又称 A 命题):所有的 $X$ 都是 $Y$;
            \item 全称否定(Universal Negative,又称 E 命题):没有 $X$ 是 $Y$;
            \item 特称肯定(Particular Affirmative,又称 I 命题):一些 $X$ 是 $Y$;
            \item 特称否定(Particular Negative,又称 O 命题):一些 $X$ 不是 $Y$。
        \end{enumerate}
        然后我们可以使用上面的运算系统表达这些逻辑命题。

        首先我们先简单思考一下如何表达“不是 $X$”。由于全集是 $1$,且“是 $X$”和“不是 $X$”加起来就是全集,因此我们可以把“不是 $X$”记作 $1 - x$。

        接着我们分别考虑这四个逻辑命题的表达方式:
        \begin{enumerate}
            \item 全称肯定“所有的 $X$ 都是 $Y$”,也就是说在 $Y$ 中选择出那些是 $X$ 的,其结果仍为 $X$,亦即 \[xy = x\],或者说
                \begin{align}
                    x(1 - y) = 0 \label{eqn:Calculations/BooleanAlgebra/OriginAndBuilding/UniversalAffirmative}
                \end{align}
            \item 全称否定“没有 $X$ 是 $Y$”,也就是说 $X$ 和 $Y$ 没有交集,从 $X$ 中选择出那些是 $Y$ 的结果是 $0$,即
                \begin{align}
                    xy = 0 \label{eqn:Calculations/BooleanAlgebra/OriginAndBuilding/UniversalNegative}
                \end{align}
            \item 特称肯定“一些 $X$ 是 $Y$”,也就是说 $X$ 和 $Y$ 有一些交集,我们任意地指定其交集为 $V$,那么可以将其表示为
                \begin{align}
                    xy = v \label{eqn:Calculations/BooleanAlgebra/OriginAndBuilding/ParticularAffirmative}
                \end{align}
            \item 特称否定“一些 $X$ 不是 $Y$”,同理可得
                \begin{align}
                    x(1 - y) = v \label{eqn:Calculations/BooleanAlgebra/OriginAndBuilding/ParticularNegative}
                \end{align}
        \end{enumerate}

        在此之后,我们可以将普通的命题转化为这一套运算系统之上的语言,并进行运算或化简。

        为了方便使用,我们再将其抽象为描述假设与推论的工具。我们将 $X$、$Y$ 等大写字母视为命题,可以描述出两种条件三段论(Conditional Syllogism):
        \begin{enumerate}
            \item 构造性(Constructive)推论:如果 $X$ 真,那么 $Y$ 真,由于 $X$ 真,所以 $Y$ 真;
            \item 破坏性(Destructive)推论:如果 $X$ 真,那么 $Y$ 真,由于 $Y$ 假,所以 $X$ 假。
        \end{enumerate}

        在这一层抽象之后,我们需要考虑的就不是各个对象及其属性,而是命题的真性,真的值为 $1$,假的值为 $0$。同时,类似的运算系统依然可以适用。例如:
        \begin{itemize}
            \item 表示命题 $X$ 为真,可以记作 $x = 1$,或者 $1 - x = 0$;
            \item 表示命题 $X$ 为假,可以记作 $x = 0$;
            \item 表示命题 $X$ 和 $Y$ 同时为真,可以记作 $xy = 1$;
            \item 表示命题 $X$ 和 $Y$ 同时为假,可以记作 $(1 - x)(1 - y) = 1$,或者 $x + y - xy = 0$;
            \item 表示命题 $X$ 和 $Y$ 一真一假,可以记作 $(1 - x)(1 - y) = 0$,或者 $x + y - xy = 1$
        \end{itemize}

        我们也可以进行一些逻辑推论,例如
        \begin{itemize}
            \item 构造性条件三段论:如果 $X$ 真,那么 $Y$ 真(于是有 $x(1 - y) = 0$),由于 $X$ 真(于是有 $x = 1$),可得 $1 - y = 0$ 也就是 $y = 1$ 或者说 $Y$ 真;
            \item 破坏性条件三段论:如果 $X$ 真,那么 $Y$ 真(于是有 $x(1 - y) = 0$),由于 $Y$ 假(于是有 $y = 0$),可得 $x = 0$ 也就是 $X$ 假。
        \end{itemize}

        事实上,我们构建出了一套取值范围只有 $0$ 和 $1$ 两种情况的数值系统,这个系统可以用于表达命题的真假,还可以表达命题间的逻辑关系。后来这个数值系统被命名为布尔代数(Boolean Algebra)。

    \subsection{定义}\label{subsec:Calculations/BooleanAlgebra/Definition}
        布尔代数是

    \subsection{运算与定律}\label{subsec:subsec:Calculations/BooleanAlgebra/OperationsAndLaws}
        布尔代数上常用的几个运算符分别称为“与”、“或”、“非”。

        \begin{itemize}
            \item “与”是一个二元运算符,其可以
            \item “或”是一个二元运算符
            \item “非”是
        \end{itemize}
