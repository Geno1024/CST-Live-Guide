\chapter*{前言}\label{ch:preface}
    \cst 是一个涵盖了丰富内容的学科。\\
    在常人认知中,“会计算机”经常与“会写程序”和“会修电脑”关联在一起,但是真正深入\cst 并不意味着止步于写一两个程序或者修一两台电脑。\cst 是一个包含了数学、逻辑、物理等学科内容的综合学科。\\
    在现代社会,大多数人的日常生活与学习已经离不开计算机。有的人只是需要使用作
