\section*{写作规范}\label{sec:Preface/WritingGuideline}
    本书使用 \hologo{XeLaTeX} 写成,格式、内容及表述遵循着一个 Geno 模仿其他排版规范而生造出来的不严谨的规范,在此列举如下:

    \subsection*{章节结构组织【组织】}\label{subsec:Preface/WritingGuideline/ChapterAndSectionOrganization}
        \begin{enumerate}
            \item 【结构】每一“章”(\textbackslash chapter)围绕一个核心主题,在一“章”中展开若干“节”(\textbackslash section),在“节”中可以展开“小节”(\textbackslash subsection, \textbackslash subsubsection, \ldots)。
            \item 【文件命名】每一节对应一个 \texttt{.tex} 文件,文件名为大驼峰(CamelCase)\footnote{即每个单词均大写首字母,且单词与单词之间没有任何间隔符或连字符。}形式的该节标题的英文翻译,并放置在以该节所在的章的大驼峰形式英文翻译的文件夹内。
            \item 【章首】每一章均有一个 \texttt{Intro.tex} 文件,描述了该章的标题及介绍。
        \end{enumerate}
    \subsection*{}