\chapter*{写作规范}\label{ch:Preface/WritingGuideline}
    本书使用 \hologo{XeLaTeX} 写成,格式、内容及表述遵循着一个 Geno 模仿其他排版规范而生造出来的不严谨的规范,在此列举如下:

    \begin{enumerate}
        \item 【文法】本书使用简体中文编写,文风将尽量与教科书类似,行文应符合现代汉语语法和语言习惯,避免欧化汉语表达。
        \item 【结构】每一“章”(\textbackslash chapter)围绕一个核心主题,在一“章”中展开若干“节”(\textbackslash section),在“节”中可以展开“小节”(\textbackslash subsection, \textbackslash subsubsection, \ldots)。
        \item 【文件命名】每一节对应一个 \texttt{.tex} 文件,文件名为大驼峰(CamelCase)\footnote{即每个单词均大写首字母,且单词与单词之间没有任何间隔符或连字符。}形式的该节标题的英文翻译,并放置在以该节所在的章的大驼峰形式英文翻译的文件夹内。
        \item 【章首】每一章均有一个 \texttt{Intro.tex} 文件,描述了该章的标题及内容摘要。
        \item 【留白】中文与英文或数字之间需要 1 个半节空格\footnote{中文文案排版指北\cite{sparanoid-cwg}}。
        \item 【概念】首次出现的概念、术语,应使用简体中文表达,并给出一个英文翻译及直观的解释。
        \item 【人名】对于国外人名,在正文中使用英文名称,在引用中使用人名原文。
        \item 【书名】对于国外书名,在正文中使用简体中文翻译,在引用中使用书名原文。
        \item 【脚注】对于附加的注释,使用 \textbackslash footnote 进行脚注,和 / 或使用 \textbackslash cite 进行引文。\footnote{本条需细化。}
    \end{enumerate}
