\section{进位制的表示与转换}\label{sec:ArithBasics/positional-notation-presentation-and-conversion}
    同一个数在不同的进位制中会有不同的表现形式,于是出现了不同进位制之间的表示与转换的问题。
    \subsection{进位制的表示}\label{subsec:ArithBasics/positional-notation-presentation-and-conversion/presentation}
        在数学上,记号通常由发明记号的数学家的想象力与其所学习的语言数量决定;而在计算机相关的方面,记号通常会受到输入输出的设备与系统的限制,因此数学和计算机(编程)方面的表示会有所不同。
        \begin{itemize}
            \item 数学上较为常见的记号是,将数直接写出,然后将底数作为下标附在其后。例如,二进制的 $1010$ 记为\[1010_2\]。有的记号则将数或基数使用小括号包裹,例如\[1010_{(2)}\]或\[(1010)_2\],这些都是常见的数学上的进位制记号。
            \item 计算机方面,在不同的场合与不同的程序语言中有不同的记号,并且大多数仅指定了记录十进制、十六进制和八进制的记号,少数指定了二进制的记号,少见可以指定任意进制的记号。最为常见的记号来自 C 语言,指定了十六进制前缀 \texttt{0x} 和八进制前缀 \texttt{0},另外还有见于颜色表示的十六进制前缀 \texttt{\#},见于 HTML 编码的 \texttt{x} 等。有时需要根据上下文才能判断具体的进制。
        \end{itemize}
    \subsection{不同进位制的转换}\label{subsec:ArithBasics/positional-notation-presentation-and-conversion/conversion}
        根据进位制的原理,我们可以计算出一个(我们所熟悉的)十进制数所对应的其他进制的表示,也可以将一个其他进制的表示转换为十进制。这样,借助我们所熟悉的十进制,我们就能做到任意进制的相互转换。若源进制与目标进制的基数在分解质因数后有一定的关系,或者非常熟悉目标进制的运算的话,我们也可以不借助十进制直接进行不同进位制的转换。

        \subsubsection{十进制转换为其他进制}\label{subsubsec:ArithBasics/positional-notation-presentation-and-conversion/conversion/from-decimal}
            将一个十进制数转换为其他进制有几种方法。整数部分可以使用短除法或者进位法,小数部分可以使用连乘法或者退位法。
            \begin{enumerate}
                \item 整数的短除法

                    首先,将这个十进制数除以目标进制的以十进制表示的基底,记录该除法结果的余数在目标进制下的表示,并将商继续进行这种除法,直到商为 0 为止,然后将记录的余数从最后到最开始的顺序书写,即为该数的目标进制的表示。

                    例如,欲将十进制整数 $13579$ 转换为十六进制:

                    \begin{enumerate}
                        \item 将该十进制整数除以目标进制的以十进制表示的基底(即 $16$),得到商 $848$,余数 $11$ 转换为目标进制的结果为 $B$;
                        \item 商 $848$ 继续除以 $16$ 得到商 $53$,余数 $0$(整除);
                        \item $53$ 除以 $16$ 得到商 $3$,余数 $5$;
                        \item $3$ 除以 $16$ 得到商 $0$,余数 $3$,短除法结束;
                        \item 将计算得到的余数从下往上读取一遍,即得到结果 $(350B)_{16}$。
                    \end{enumerate}

                    % https://tex.stackexchange.com/a/210652/149813
                    \begin{figure}
                        \centering
                        \begin{tabular}{rrlll}
                            16 \shortdiv{13579} &    &                 &   &                        \\
                            16   \shortdiv{848} & 11 & \textrightarrow & B & \tikzmark{sdi-end}     \\
                            16    \shortdiv{53} &  0 & \textrightarrow & 0 &                        \\
                            16     \shortdiv{3} &  5 & \textrightarrow & 5 &                        \\
                            0                   &  3 & \textrightarrow & 3 & \tikzmark{sdi-start}
                        \end{tabular}
                        \begin{tikzpicture}[overlay, remember picture]
                            \draw[->] ($(pic cs:sdi-start)+(0pt,.5ex)$) to ($(pic cs:sdi-end)+(0pt,.5ex)$);
                        \end{tikzpicture}
                        \caption{短除法转换十进制整数为其他进制的示例}
                        \label{fig:ArithBasics/positional-notation-presentation-and-conversion/conversion/from-decimal/short-division-integer}
                    \end{figure}
                \item 小数的连乘法

                    将小数部分乘以目标进制的以十进制表示的基底,记录该乘法结果的整数部分在目标进制下的表示,并将小数部分继续进行这种乘法,直到小数部分为 0 为止,然后将记录的整数部分从最开始到最后的顺序书写,即为该数的目标进制的表示。

                    注意,将小数进行进制转换时,很容易出现“死循环”,即永远无法达到“小数部分为 0”的情况。这是非常常见的情况,此时转换的结果会是一个无限循环小数或者无限不循环小数,就像 $\frac{1}{3}$ 在十进制下表示为无限循环小数 $0.\overline{3}$ 而在三进制下简单地表示为 $0.1$ 一样。

                    例如,欲将十进制小数 $0.3125$ 转换为二进制:
                    \begin{enumerate}
                        \item 将小数部分乘以目标禁止的以十进制表示的基底(即 $2$),得到 $0.625$,整数部分为 $0$,小数部分为 $.625$;
                        \item 小数部分 $.625$ 继续乘以 $2$ 得到 $1.25$,整数部分为 $1$,小数部分为 $.25$;
                        \item $.25$ 乘以 $2$ 得到 $0.5$,整数部分为 $0$,小数部分为 $.5$;
                        \item $.5$ 乘以 $2$ 得到 $1$,整数部分为 $1$,小数部分为 $.$,连乘法结束;
                        \item 将计算得到的整数部分从上往下读取一遍,即得到结果 $(0.0101)_2$。
                    \end{enumerate}

                    \begin{figure}
                        \centering
                        \begin{tabular}{llll}
                            \tikzmark{lmf-start}    &   &          & .3125 \\
                                                    &   & $\times$ &     2 \\ \hline
                                                    & 0 &          & .625  \\
                                                    &   & $\times$ &    2  \\ \hline
                                                    & 1 &          & .25   \\
                                                    &   & $\times$ &   2   \\ \hline
                                                    & 0 &          & .5    \\
                                                    &   & $\times$ &  2    \\ \hline
                            \tikzmark{lmf-end}      & 1 &          & .
                        \end{tabular}
                        \begin{tikzpicture}[overlay, remember picture]
                            \draw[->] ($(pic cs:lmf-start)+(0pt,.5ex)$) to ($(pic cs:lmf-end)+(0pt,.5ex)$);
                        \end{tikzpicture}
                        \caption{连乘法转换十进制小数为其他进制的示例}
                        \label{fig:ArithBasics/positional-notation-presentation-and-conversion/conversion/from-decimal/long-multiplication-fractal}
                    \end{figure}

                \item
            \end{enumerate}
        \subsubsection{其他进制转换为十进制}\label{subsubsec:ArithBasics/positional-notation-presentation-and-conversion/conversion/to-decimal}
            将一个非十进制的数转换为十进制,
