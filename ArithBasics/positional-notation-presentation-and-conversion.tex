\section{进位制的表示与转换}\label{sec:ArithBasics/positional-notation-presentation-and-conversion}
    同一个数在不同的进位制中会有不同的表现形式,于是出现了不同进位制之间的表示与转换的问题。
    \subsection{进位制的表示}\label{subsec:ArithBasics/positional-notation-presentation-and-conversion/presentation}
        在数学上,记号通常由发明记号的数学家的想象力与其所学习的语言数量决定;而在计算机相关的方面,记号通常会受到输入输出的设备与系统的限制,因此数学和计算机(编程)方面的表示会有所不同。
        \begin{itemize}
            \item 数学上较为常见的记号是,将数直接写出,然后将底数作为下标附在其后。例如,二进制的 1010 记为\[1010_2\]。有的记号则将数或基数使用小括号包裹,例如\[1010_{(2)}\]或\[(1010)_2\],这些都是常见的数学上的进位制记号。
            \item 计算机方面,在不同的场合与不同的程序语言中有不同的记号,并且大多数仅指定了记录十进制、十六进制和八进制的记号,少数指定了二进制的记号,少见可以指定任意进制的记号。最为常见的记号来自 C 语言,指定了十六进制前缀 \texttt{0x} 和八进制前缀 \texttt{0},
            \item
        \end{itemize}
