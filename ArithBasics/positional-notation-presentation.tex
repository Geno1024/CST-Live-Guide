\section{进位制的表示}\label{sec:ArithBasics/positional-notation-presentation}
    同一个数在不同的进位制中会有不同的表现形式,于是出现了不同进位制之间的表示与转换的问题。
    \subsection{记号表示}\label{subsec:ArithBasics/positional-notation-presentation/natural}
        在数学上,记号通常由发明记号的数学家的想象力与其所学习的语言数量决定;而在计算机相关的方面,记号通常会受到输入输出的设备与系统的限制,因此数学和计算机(编程)方面的表示会有所不同。
        \begin{itemize}
            \item 数学上较为常见的记号是,将数直接写出,然后将底数作为下标附在其后。例如,二进制的 $1010$ 记为\[1010_2\]。有的记号则将数或基数使用小括号包裹,例如\[1010_{(2)}\]或\[(1010)_2\],这些都是常见的数学上的进位制记号。
            \item 程序语言方面,在不同的场合与不同的程序语言中有不同的记号,并且大多数仅指定了记录十进制、十六进制和八进制的记号,少数指定了二进制的记号,少见可以指定任意进制的记号。最为常见的记号来自 C 语言,指定了十六进制前缀 \texttt{0x} 和八进制前缀 \texttt{0},另外还有见于颜色表示的十六进制前缀 \texttt{\#},见于 HTML 编码的 \texttt{x} 等。有时需要根据上下文才能判断具体的进制。
        \end{itemize}
    \subsection{二进制表示}\label{subsec:ArithBasics/positional-notation-presentation/binary}
        由于现代计算机均采用二进制,因此需要找到一种可以只用二进制表示各种数,并且符合基本运算法则的方式。另外,由于存储空间的限制,我们需要考虑一个有限长度的记录方式,而非像数学那样理论上可以无限添加长度的记录方式(因此,这些记录方式通常存在一个能表达的数的范围,称为“数值范围”。关于数值范围,以及超出数值范围的数的表达,将在后续章节中再描述,此处暂时忽略)。

        为了讨论方便和章节简短,在本小节中,我们先考虑一个 $8$ 位的存储整数的设备(即,必须而且只能使用 $8$ 位二进制来表示,若超过 $8$ 位则将其截断,只保留最右边的 $8$ 位)。

        最容易想到的记录方式为,直接记录将源数字转换为二进制之后的结果,比如说将 $1$ 表达为 $0000\ 0001$,把 $2$ 表达为 $0000\ 0010$ 等,但是此种方式无法记录负数。为了顺利地记录负数,几种记录方式被制造了出来。

        \subsubsection{原码}\label{subsubsec:ArithBasics/positional-notation-presentation/binary/sign-magnitude}
            原码(Sign-Magnitude Form)指定,
            \begin{itemize}
                \item 最高位为符号位(Sign),并任意地指定 $0$ 表示正数,$1$ 表示负数;
                \item 剩下的位为数值位(Magnitude),记录准备表达的数的绝对值。
            \end{itemize}
            例如,$-1$ 记录为 $1000\ 0001$,$-2$ 记录为 $1000\ 0010$,$-127$ 记录为 $1111\ 1111$,等等。这个方式能够表达负数,但是轻易尝试将发现问题:
            \[1000\ 0001 + 0000\ 0001 = 1000\ 0010\],
            这对应于十进制数将是,$(-1) + 1 = (-2)$,是非常荒唐的。因此需要继续寻找其他的记录方式。

        \subsubsection{一补码与反码}\label{subsubsec:ArithBasics/positional-notation-presentation/binary/one-complement}
            一补码(1's Complement)指定,对于一个正数的相反数,表达为该数二进制表达的各位取反(即,$1$ 变为 $0$,$0$ 变为 $1$)。
            反码指定,当表达正数时,使用数的二进制表达;当表达负数时,使用一补码,即
            \begin{itemize}
                \item 最高位为符号位,$0$ 表示正数,$1$ 表示负数;
                \item 剩下的位为数值位,当表示正数时直接记录准备表达的数的二进制,当表示负数时记录准备表达的数的二进制的各位取反。
            \end{itemize}
