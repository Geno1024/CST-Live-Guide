\section{进位制}\label{sec:ArithBasics/positional-notation}
    在原始社会中,计数与记数的方式伴随着生产与劳动成果分配的过程而萌芽、发展。最原始的记数方式包括了石子记数、结绳记数等。由于进化的偶然,包括人类在内的灵长类动物均具有两只手,每只手上有五根手指。这被普遍认为是十进制的生物学起源。\\
    \subsection{十进制}\label{subsec:ArithBasics/positional-notation/decimal}
        从我们学习数数开始,我们便耳濡目染地接触了十进制。我们使用 \texttt{1} 到 \texttt{9} 这九个数码以表征一个数字,并且当要表征的数字比 \texttt{9} 大的时候,往“前”“进一”并使用数码 \texttt{0} 进行占位。这个“进”导致相同的数码在不同的位置上表示的意义不同。例如:
        \[100\]
        的 \texttt{1} 表示一个“百”,而
        \[10000\]
        的 \texttt{1} 表示一个“万”。在进位制中,不同的位置上的数码,要乘以它所在的位的“权重”。对于十进制而言,这个“权重”就是我们通常所说的“个”、“十”、“百”、“千”、“万”等。
        \begin{displayquote}
            汉语中对于更大的数字的权重,传统上有亿、兆、京、垓、秭、穰、沟、涧、正、载十个名称。但对于这些名称具体表达的权重大小存在不同的定义,这也造成了现代汉语表达上的混淆。\\
            \begin{itemize}
                \item 下数:“十十变之”\footnote{\footfullcite{wujin}\cite{wujin}},也就是每十倍即使用下一个权重名称,即 10 万为 1 亿($10^6$),10 亿为 1 兆($10^7$),10 兆为 1 京($10^8$),依此类推。
                \item 中数:“万万变之”\footnote{\footfullcite{wujin}\cite{wujin}},也就是每万万倍即使用下一个权重名称,即 1 万万为 1 亿($10^8$),1 万万亿为 1 兆($10^{16}$),1 万万兆为 1 京($10^{24}$),依此类推。
                \item 上数:“数穷则变”\footnote{\footfullcite{wujin}\cite{wujin}},也就是递归地使用更小的权重名称,直到用完才使用更大的权重名称,即 1 万万为 1 亿($10^8$),1 亿亿为 1 兆($10^{16}$),1 兆兆为 1 京($10^{32}$),依此类推。
            \end{itemize}
        \end{displayquote}
