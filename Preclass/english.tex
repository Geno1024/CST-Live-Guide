\section{英文能力}\label{sec:Preclass/english}
    由于一些历史渊源,英语确立了国际通用交流语言的地位,有大量的技术文档与标准、科研论文、报错日志、提示信息等均使用英文撰写。因此,英文能力是学习计算机,特别是深入学习计算机所必备的技能。

    现代意义上的电子计算机一开始是美国在二战时期为了计算导弹弹道而制作的巨型计算器。美国和前苏联在二战和冷战期间均努力发展计算机,但美国发展速度更快,且抢先一步发展出了互联网,而我国大规模引入计算机及互联网时间均较晚。另外,作为二战与冷战时期的产物,早期的计算机硬件、操作系统及软件等在人机交互方面几乎仅考虑了制造国的通用语言,与语言相关的规定及限制甚至成为了标准的一部分。而为了让较老旧的设备也可以继续使用,一些新的标准会选择兼容旧标准,这导致了早期的带有语言限制的标准可能流传到现在。例如 ASCII 作为一个经典的通用编码只考虑了英文字母,甚至不包含一些拉丁字母的西欧变体,这导致大量的计算机技术文档、论文与标准、程序命令与操作系统提示等均使用了英文。

    种种历史结合的结果造成了现在的局面:英语成为事实上的国际通用交流语言及互联网上的通用语。因此,如果要深入研究计算机的历史或者前沿方向,流畅的英语阅读能力是必不可少的技能。
