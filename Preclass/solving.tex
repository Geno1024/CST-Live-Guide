\section{科学解决问题}\label{sec:solving}
    在学习与生活的过程中,必然会遇到大量的问题。以计算机为例,从软件安装到环境配置,从代码运行到漏洞利用,从买卡装机到硬件开发,无时无刻不充满着大量的技巧、规范与约定。一旦对某个知识点不甚了解,就可能会出现大量的问题。

    当遇到一个问题之后,我们将会选择放弃,或者努力地解决它。放弃会非常轻松,但是不断地往正确方向努力可以让你走向成功。本节将试图描述一些解决问题过程中的科学方法。

    本文作者认为,“科学解决问题”包含了两个步骤,一个是问题的提出,一个是问题的解答。

    \subsection{提问的智慧}\label{subsec:solving/asking}
        本节部分或全部借鉴自 Eric S. Raymond 的《提问的智慧》\cite{esr},剩余的来自 Geno 的亲身实践。

        总而言之,一个合格的问题,应至少包含以下几个部分:
        \begin{enumerate}
            \item 目标:“我准备做什么”;
            \item 预期:“我希望看到什么”;
            \item 现象:“我实际看到了什么”;
            \item 尝试:“我试着做过什么(但并没有效果(不然也不需要提问题了))”。
        \end{enumerate}

    \subsection{回答的技巧}\label{subsec:solving/answering}
